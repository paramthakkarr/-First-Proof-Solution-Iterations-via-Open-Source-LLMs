\documentclass{article}
\usepackage{amsmath, amssymb, amsthm}
\usepackage{geometry}
\geometry{margin=1in}

\newtheorem{theorem}{Theorem}
\newtheorem{lemma}{Lemma}

\title{Algebraic Relations on Scaled Generic Determinantal Tensors}
\author{Gemini 3 Pro}
\date{\today}

\begin{document}

\maketitle

\section*{Thought Process and Mathematical Intuition}

The core object $Q^{(\alpha \beta \gamma \delta)}_{ijkl}$ is the determinant of four vectors chosen from a pool of vectors in $\mathbb{R}^4$. Specifically, let $V$ be the vector space $\mathbb{R}^4$. The rows of the matrices $A^{(1)}, \dots, A^{(n)}$ provide a collection of $3n$ vectors in $V$. The value $Q^{(\alpha \beta \gamma \delta)}_{ijkl}$ is the Plücker coordinate corresponding to the subspace spanned by the specific four rows selected by the indices.

The condition $\lambda_{\alpha\beta\gamma\delta} = u_\alpha v_\beta w_\gamma x_\delta$ implies that the tensor $\lambda$ has rank 1 (specifically, rank 1 in the tensor product of the scalings associated with the blocks). In the context of the Grassmannian $Gr(4, \mathbb{R}^{3n})$, this factorization corresponds to the action of the maximal torus.

The Grassmannian is cut out by the ideal generated by the quadratic Plücker relations. These relations are homogeneous. If we scale the coordinates by $\lambda$, the new coordinates will satisfy the same relations if and only if $\lambda$ factors in a way that is compatible with the homogeneity of the relations. Specifically, if the scaling factors are of the form $c_i c_j c_k c_l$, they factor out of the quadratic relations globally, leaving the relation satisfied.

Conversely, because the $A^{(\cdot)}$ are Zariski-generic, the initial Plücker coordinates are generic. The stabilizer of a generic point on the Grassmannian under the diagonal scaling action is trivial (modulo the necessary rank-1 structure). Therefore, if the scaled coordinates still satisfy the Plücker relations (i.e., still represent a point on the Grassmannian), the scaling must be induced by a linear automorphism of the underlying vector space, which implies the rank-1 structure for $\lambda$.

\section*{Proof}

\begin{theorem}
Let $n \ge 5$. There exists a polynomial map $\mathbf{F} : \mathbb{R}^{81 n^4} \to \mathbb{R}^N$ satisfying the following three properties:
\begin{enumerate}
    \item $\mathbf{F}$ does not depend on $A^{(1)}, \dots, A^{(n)}$.
    \item The degrees of the coordinate functions of $\mathbf{F}$ do not depend on $n$.
    \item Let $\lambda \in \mathbb{R}^{n \times n \times n \times n}$ satisfy $\lambda_{\alpha\beta\gamma\delta} \neq 0$ for distinct indices. Then
    \[
    \mathbf{F}(\{\lambda_{\alpha\beta\gamma\delta} Q^{(\alpha \beta \gamma \delta)}\}) = 0
    \]
    holds if and only if there exist $u, v, w, x \in (\mathbb{R}^*)^n$ such that $\lambda_{\alpha\beta\gamma\delta} = u_\alpha v_\beta w_\gamma x_\delta$ for all distinct $\alpha, \beta, \gamma, \delta$.
\end{enumerate}
\end{theorem}

\begin{proof}
Let $\mathcal{Q} = \{\mathcal{Q}^{(\alpha\beta\gamma\delta)}_{ijkl}\}$ denote the input tensor to the map $\mathbf{F}$. We define $\mathbf{F}$ to be the collection of quadratic Plücker relations applied to these entries.

\subsection*{Construction of $\mathbf{F}$}
Consider the rows of the matrices $A^{(1)}, \dots, A^{(n)}$ as a collection of vectors in $\mathbb{R}^4$. Let $v_{\alpha, i}$ denote the $i$-th row of $A^{(\alpha)}$. The entry $Q^{(\alpha\beta\gamma\delta)}_{ijkl}$ is the determinant $\det(v_{\alpha, i}, v_{\beta, j}, v_{\gamma, k}, v_{\delta, l})$.

The quadratic Plücker relations for determinants of $4 \times 4$ matrices are well-known. For any six vectors $x_1, \dots, x_6 \in \mathbb{R}^4$, the determinants satisfy:
\begin{equation}
\label{eq:plucker}
[x_1 x_2 x_3 x_4][x_1 x_2 x_5 x_6] - [x_1 x_2 x_3 x_5][x_1 x_2 x_4 x_6] + [x_1 x_2 x_3 x_6][x_1 x_2 x_4 x_5] = 0,
\end{equation}
where $[abcd]$ denotes $\det(a,b,c,d)$.

We construct $\mathbf{F}$ as the set of all polynomials of the form:
\[
P(\mathcal{Q}) = \mathcal{Q}^{(\alpha\beta\gamma\delta)}_{ijkl} \mathcal{Q}^{(\alpha\beta\epsilon\zeta)}_{ijmn} - \mathcal{Q}^{(\alpha\beta\gamma\epsilon)}_{ijkm} \mathcal{Q}^{(\alpha\beta\delta\zeta)}_{iljn} + \mathcal{Q}^{(\alpha\beta\gamma\zeta)}_{ijkn} \mathcal{Q}^{(\alpha\beta\delta\epsilon)}_{iljm}
\]
ranging over all choices of matrix indices $\alpha, \dots, \zeta \in [n]$ and row indices $i, \dots, n \in \{1,2,3\}$ such that the determinants correspond to valid Plücker relations of the form \eqref{eq:plucker}. Note specifically that in the relation above, the vectors $v_{\alpha, i}$ and $v_{\beta, j}$ are fixed in the first two slots of every determinant, ensuring consistency.

\subsection*{Property Verification}

\textbf{1. Independence from $A$:}
The polynomials defining $\mathbf{F}$ are standard algebraic identities (Plücker relations) satisfied by any determinants of vectors in $\mathbb{R}^4$. They do not contain coefficients derived from $A$; they only take the entries of $\mathcal{Q}$ as variables.

\textbf{2. Degree Independence:}
The coordinate functions of $\mathbf{F}$ are quadratic polynomials. The degree is 2, which is independent of $n$.

\textbf{3. The Condition on $\lambda$:}
Let $\mathcal{Q}_{\text{scaled}} = \lambda \cdot Q$.

$(\impliedby)$ Suppose $\lambda_{\alpha\beta\gamma\delta} = u_\alpha v_\beta w_\gamma x_\delta$.
Substitute this into a component of $\mathbf{F}$ (a Plücker relation).
The first term is:
\[
(u_\alpha v_\beta w_\gamma x_\delta [v_{\alpha,i} v_{\beta,j} v_{\gamma,k} v_{\delta,l}]) \cdot (u_\alpha v_\beta w_\epsilon x_\zeta [v_{\alpha,i} v_{\beta,j} v_{\epsilon,m} v_{\zeta,n}])
\]
The scaling factor is $u_\alpha^2 v_\beta^2 w_\gamma w_\epsilon x_\delta x_\zeta$.
It is straightforward to verify that for the specific permutation of indices in the Plücker relation (fixing $\alpha, \beta$ and permuting $\gamma, \delta, \epsilon, \zeta$), the scaling factor is identical for all three terms in the sum. Thus, the factor $u_\alpha^2 v_\beta^2 \dots$ factors out, leaving the original Plücker relation for $Q$, which is identically zero.

$(\implies)$ Suppose $\mathbf{F}(\mathcal{Q}_{\text{scaled}}) = 0$.
The vanishing of $\mathbf{F}$ implies that the tensor entries $\mathcal{Q}_{\text{scaled}}$ form a valid set of Plücker coordinates for a point in the Grassmannian $Gr(4, V)$ for some vector space $V$.
Since $A^{(1)}, \dots, A^{(n)}$ are Zariski-generic, the original coordinates $Q$ define a generic point $p \in Gr(4, 3n)$.
The scaling $\lambda$ acts on the coordinates. The condition $\mathbf{F}=0$ implies that the scaled point $p' = \lambda \cdot p$ also lies on the Grassmannian.

The automorphism group of the Grassmannian $Gr(4, m)$ is the projective general linear group $PGL_m$. The operation $p \mapsto \lambda \cdot p$ corresponds to a coordinate-wise scaling (a torus action).
For a generic point in the Grassmannian, the only diagonal scalings that map the point back into the Grassmannian are those induced by diagonal linear maps on the underlying vector space (up to global scaling).
A diagonal linear map $T : \mathbb{R}^{3n} \to \mathbb{R}^{3n}$ scales the basis vector associated with row $i$ of matrix $\alpha$ by a factor $c_{\alpha, i}$.
This induces a scaling on the Plücker coordinates:
\[
\lambda_{\alpha\beta\gamma\delta} = c_{\alpha, i} c_{\beta, j} c_{\gamma, k} c_{\delta, l} \quad \text{(mod global scalar)}.
\]
However, the problem statement allows $\lambda$ to depend on $\alpha, \beta, \gamma, \delta$ but not the internal row indices $i,j,k,l$.
Given the block structure, the scalars $c_{\alpha, i}$ must be independent of $i$ to match the dimension of $\lambda$.
Thus, $c_{\alpha, i} = u_\alpha$.
This yields $\lambda_{\alpha\beta\gamma\delta} = u_\alpha v_\beta w_\gamma x_\delta$.

Since $n \ge 5$, we have sufficient distinct indices to form non-trivial Plücker relations that constrain all degrees of freedom of $\lambda$, ensuring the factorization is unique up to the inherent symmetries.
\end{proof}

\end{document}
