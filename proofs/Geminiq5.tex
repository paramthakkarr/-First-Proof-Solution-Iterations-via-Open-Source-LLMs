\documentclass[11pt]{article}
\usepackage[utf8]{inputenc}
\usepackage[T1]{fontenc}
\usepackage{amsmath, amssymb, amsthm}
\usepackage{mathrsfs}
\usepackage{tikz-cd}
\usepackage{geometry}
\geometry{a4paper, margin=1in}

% Theorem Environments
\newtheorem{theorem}{Theorem}[section]
\newtheorem{lemma}[theorem]{Lemma}
\newtheorem{proposition}[theorem]{Proposition}
\newtheorem{corollary}[theorem]{Corollary}

\theoremstyle{definition}
\newtheorem{definition}[theorem]{Definition}
\newtheorem{remark}[theorem]{Remark}

% Macros
\newcommand{\G}{G}
\newcommand{\SpG}{\text{Sp}^G}
\newcommand{\Osys}{\mathcal{O}}
\newcommand{\PhiH}{\Phi^H}
\newcommand{\taugen}{\tau_{\ge n}^{\Osys}}
\newcommand{\slice}[1]{P_{#1}}

\title{Characterization of $\Osys$-Slice Connectivity via Geometric Fixed Points}
\author{}
\date{}

\begin{document}

\maketitle

\section*{The $\Osys$-Slice Connectivity Theorem}

Let $G$ be a finite group and let $\Osys$ denote an incomplete transfer system associated to an $N_\infty$ operad. We consider the slice filtration on the $G$-equivariant stable category $\SpG$ adapted to $\Osys$. Let $\tau_{\ge n}^{\Osys}$ denote the localizing subcategory of $n$-slice connected objects.

The following theorem characterizes membership in the $n$-th slice filtration via the connectivity of geometric fixed points. Note that in the standard slice filtration (Hill-Hopkins-Ravenel), a slice cell of dimension $d$ typically contributes to the $H$-geometric fixed points in dimension $d/|H|$.

\begin{theorem}
Let $X$ be a connective $G$-spectrum. Then $X \in \taugen$ if and only if for all subgroups $H \le G$ (compatible with the indexing system $\Osys$), the geometric fixed points $\PhiH X$ satisfy the connectivity condition:
\[
\pi_k(\PhiH X) = 0 \quad \text{for all } k < \frac{n}{|H|}.
\]
\end{theorem}

\begin{proof}
Let $\mathcal{C}_{\ge n}$ denote the class of connective $G$-spectra satisfying the condition that $\pi_k(\PhiH X) = 0$ for $k < n/|H|$ for all relevant subgroups $H$. We wish to show $\taugen = \mathcal{C}_{\ge n}$.

\medskip
\noindent \textbf{Necessity ($\subseteq$):}
The category $\taugen$ is the localizing subcategory generated by a set of slice cells $\mathcal{G}_n$. These generators typically take the form $\hat{S} = G_+ \wedge_K S^{m\rho_K - \epsilon}$, where the slice dimension is $d = m|K| - \epsilon \ge n$. 
Recall that the geometric fixed points functor $\PhiH: \SpG \to \text{Sp}$ is strong symmetric monoidal and preserves homotopy colimits. Therefore, it suffices to verify the condition on the generators.
For a generator $S \in \mathcal{G}_n$ of dimension $d \ge n$, we have:
\[
\PhiH(S) \simeq \bigvee_{gK \in (G/K)^H} S^{V^g}
\]
where $V$ is the underlying representation. For regular representation spheres, $\dim(V^g) = \dim(V)/|H|$. Thus, the connectivity of $\PhiH(S)$ is determined by $d/|H| \ge n/|H|$. Consequently, $\PhiH(S)$ is $(n/|H|-1)$-connected. Since $\mathcal{C}_{\ge n}$ is closed under colimits and extensions, $\taugen \subseteq \mathcal{C}_{\ge n}$.

\medskip
\noindent \textbf{Sufficiency ($\supseteq$):}
Suppose $X \in \mathcal{C}_{\ge n}$. We utilize the slice tower $\{ \slice{q} X \}$. Consider the map to the $n$-slice cover $\phi: \slice{n} X \to X$. Let $F$ be the fiber of this map, such that we have a fiber sequence:
\[
F \to \slice{n} X \to X
\]
By construction, $F \in \tau_{< n}^{\Osys}$. The geometric fixed points functor detects the slice filtration range; specifically, if $Y \in \tau_{< n}^{\Osys}$, then $\PhiH Y$ is concentrated in dimensions strictly less than $n/|H|$.

By the hypothesis on $X$, $\pi_k(\PhiH X) = 0$ for $k < n/|H|$. Since $\slice{n} X \in \taugen$, we also have $\pi_k(\PhiH \slice{n} X) = 0$ for $k < n/|H|$.
This implies that in the long exact sequence of homotopy groups for the geometric fixed points:
\[
\dots \to \pi_k(\PhiH \slice{n} X) \to \pi_k(\PhiH X) \to \pi_{k-1}(\PhiH F) \to \dots
\]
the term $\PhiH F$ must be contractible. A fundamental result in equivariant stable homotopy theory states that the collection of functors $\{ \PhiH \}_{H \le G}$ is jointly conservative on connective $G$-spectra (the detection theorem). Since $\PhiH F \simeq *$ for all $H$, we conclude $F \simeq *$.

Therefore, the map $\slice{n} X \to X$ is an equivalence, implying $X \in \taugen$.
\end{proof}

\end{document}
